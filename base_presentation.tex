\documentclass{beamer}

\usepackage[utf8]{inputenc}
\usepackage[T1]{fontenc}
\usepackage[english,brazil]{babel}
\usepackage{multicol}
\usepackage{float}
\usepackage{graphicx}
\usepackage[labelformat=empty]{caption}
%\usepackage{enumitem}

% for icons
\usepackage{fontawesome}

% for code snippets
\usepackage{listings}

% usando tema personalizado. 
% arquivo beamerthemeIFSC.sty deve estar no mesmo diretório do .tex
\usetheme{ifsc}


\hypersetup{pdfstartview={Fit},pdftitle={Aula SEP: Transmissão e Distribuição de Energia},
 	pdfsubject={Técnico em Eletrotécnica},pdfauthor={Arthur G. Bartsch}
}
\newcommand{\subitem}[1]{\begin{itemize}
\item #1
\end{itemize}}


%%%%%%%%%%%%%%%%%%%%%%%%%%%%%%%%%%%%%%%%%%%%


\title{Programabilidade de redes com eBPF no contexto de IoT}
\subtitle{Projeto de TCC}
\author{Gabriel M. de Melo}
\date{2019/02}
\institute{
Universidade Federal de Lavras\\
Lavras - MG \\
~\\
GCC220 - Metodologia de Pesquisa\\
~\\
\url{gabrielmarquesm@estudante.ufla.br}\\
}



%%%%%%%%%%%%%%%%%%%%%%%%%%%%%%%%%%%%%%%%%%%%

\begin{document}

\begin{frame}[t]
	\maketitle
\end{frame}

\frame{\frametitle{Sumário}
\tableofcontents}

\def\sectionname{}
\def\insertsectionnumber{}
\def\subsectionname{}
\def\insertsubsectionnumber{}



%\setbeamertemplate{section page}
%{
%    \begin{centering}
%    \begin{beamercolorbox}[sep=12pt,center]{part title}
%    \usebeamerfont{section title}\insertsection\par
%    \end{beamercolorbox}
%
%    \end{centering}
%}

\AtBeginSection{\frame{\sectionpage}\addtocounter{framenumber}{-1} }% \vspace{1cm}
%\tableofcontents[hideothersubsections,sectionstyle=hide/hide,subsectionstyle=hide/show]}}

%\AtBeginSection{\frame{\sectionpage \vspace{1cm}}
%\tableofcontents[currentsubsection,hideothersubsections,sectionstyle=show/hide,subsectionstyle=show/shaded] 
%}

\AtBeginSubsection{\frame{\subsectionpage}\addtocounter{framenumber}{-1} }
\AtBeginSubsubsection{\frame{\subsubsectionpage}\addtocounter{framenumber}{-1} }

%%%%%%%%%%%%%%%%%%%%%%%%%%%%%%%%%%%%%%%%%%%%
% Inicio do documento
%%%%%%%%%%%%%%%%%%%%%%%%%%%%%%%%%%%%%%%%%%%%

%\section{Listas}
%
%
%\begin{frame}{Apenas começando}
%	\begin{itemize}
%		\item Primeiro item
%		\item Segundo item
%		\item Terceiro item 
%	\end{itemize}
%	\begin{enumerate}
%		\item Primeiro item
%		\item Segundo item
%		\item Terceiro item 
%	\end{enumerate}
%\end{frame}
%
%\subsection{Blocos}
%
%
%\begin{frame}{Blocos}
%	\begin{block}{Esse é um bloco}
%		Isso é um teste
%	\end{block}
%	\begin{block}{}
%	Bloco sem título	
%	\end{block}
%	\begin{alertblock}{Alerta}
%		Esse é um alerta
%	\end{alertblock}
%\end{frame}

%
%\begin{frame}[fragile]{Código em C e Java}
%\ansic
%	\begin{lstlisting}
%	 int main(void){
%	    printf("Ola mundo\n");
%	    return 0;
%	 }
%	\end{lstlisting}
%\java
%	\begin{lstlisting}
%	 public static voi main(String args[]){
%	    System.out.println("Ola mundo");
%	 }
%	\end{lstlisting}	
%\end{frame}


\defverbatim[colored]\lstI{
\begin{lstlisting}[language=C++,basicstyle=\ttfamily,keywordstyle=\color{red}]
#include <linux/bpf.h>

int prog(struct xdp_md *ctx){
    return XDP_DROP;  // descarta todos pacotes
}
\end{lstlisting}
}


% Dados referentes ao número de usuários da internet
\section{Contextualização}
\begin{frame}{}
	\begin{figure}
		%\caption{Número de usuários ativos em 2019}
		\includegraphics[width = 0.85\linewidth]{img/proj_tcc/statistic_id617136_worldwide-digital-population-as-of-april-2019.png}\\
		\tiny Fonte: We are Social; Data Reportal; Hootsuite
	\end{figure}
\end{frame}

% Modelos de negócios dependem cada vez mais da rede e demandam conexões de menor latência e maior segurança
\begin{frame}{}
	\begin{figure}
        \centering
        \begin{minipage}{.5\textwidth}
          \centering
          \includegraphics[width=.5\linewidth]{img/proj_tcc/uber.jpg}
          \captionof{figure}{Transporte}
          \label{figure:facebook}
        \end{minipage}%
        \begin{minipage}{.5\textwidth}
          \centering
          \vspace{15pt}  % Just for alignment
          \includegraphics[width=.7\linewidth]{img/proj_tcc/netflix.png}
          \captionof{figure}{Filmes}
          \label{figure:cloudflare}
        \end{minipage}
    \end{figure}	
    \begin{figure}
        \centering
        \begin{minipage}{.5\textwidth}
          \centering
          \includegraphics[width=.5\linewidth]{img/proj_tcc/whatsapp.png}
          \captionof{figure}{Comunicação}
          \label{figure:facebook}
        \end{minipage}%
        \begin{minipage}{.5\textwidth}
          \centering
          %\vspace{10pt}  % Just for alignment
          \includegraphics[width=.5\linewidth]{img/proj_tcc/bitcoin.png}
          \captionof{figure}{Moeda}
          \label{figure:cloudflare}
        \end{minipage}
    \end{figure}
\end{frame}

% No caso de redes IoT, onde muitos dispositivos trocam mensagens (que muitas vezes carregam informações sensíveis) entre si e com a internet, essa realidade é ainda mais complicada.
\begin{frame}{}
	\begin{figure}
		\caption{}
		\includegraphics[width = 0.8\linewidth]{img/proj_tcc/iot_network.jpg}\\
		\tiny Fonte: VectoMobile
	\end{figure}
\end{frame}

% Surge a necessidade de dinamizar a rede inserindo programabilidade e a tornando flexível e mais segura.
\begin{frame}{Programabilidade de redes}
	\begin{itemize}
	    \item Dinamicidade e Flexibilidade
	    \item Gerência facilitada
		\item Segurança (remediação e predição)
		\end{itemize} 
\end{frame}


% eBPF + imagem do facebook e Cloudfare
\section{eBPF (extended Berkeley Packet Filter)}
\begin{frame}{eBPF - extended Berkeley Packet Filter}
\begin{itemize}
		\item Máquina virtual no kernel (\textit{segurança}).
		\item Interface dinâmica entre \textit{user space} e \textit{kernel space}.
		\item Alternativa flexível aos módulos atuais do kernel.
	\end{itemize} 
\begin{figure}
\centering
\begin{minipage}{.5\textwidth}
  \centering
  \includegraphics[width=.4\linewidth]{img/proj_tcc/facebook-logo.png}
  \captionof{figure}{Balanceador de carga}
  \label{figure:facebook}
\end{minipage}%
\begin{minipage}{.5\textwidth}
  \centering
  \vspace{28pt}  % Just for alignment
  \includegraphics[width=.6\linewidth]{img/proj_tcc/cloudflare.png}
  \captionof{figure}{Mitigação de ataques}
  \label{figure:cloudflare}
\end{minipage}
\end{figure}	
\end{frame}

% imagem com arquitetura da vm e workflow (passos e maps)
\begin{frame}{eBPF - extended Berkeley Packet Filter}
    \begin{figure}
        \centering
        \includegraphics[width=.55\linewidth]{img/proj_tcc/maquina-ebpf.png}
        \label{figure:facebook}
    \end{figure}
    \begin{figure}
      \centering
      \includegraphics[width=.8\linewidth]{img/proj_tcc/workflow.png}
      \label{figure:cloudflare}
    \end{figure}
\end{frame}

% empregos do eBPF
\begin{frame}{eBPF - extended Berkeley Packet Filter}
    \begin{itemize}
        \item kprobes (uprobes, kretprobes, uretprobes)
        \item socket filter
        
        ...
        
        \item XDP (\textit{eXpress Data Path})
    \end{itemize}
\end{frame}

% eBPF e XDP. Figura da pilha de rede do kernel e explicação do funcionamento
\begin{frame}{eBPF \& XDP (\textit{eXpress Data Path})}
	\begin{figure}
		\caption{}
		\includegraphics[width = 0.65\linewidth]{img/proj_tcc/kernel-network-stack.png}\\
		\tiny Fonte: HootSuite, Abril de 2019
	\end{figure}
\end{frame}

% código simples que realiza o descarte de todos os pacotes

\begin{frame}{eBPF \& XDP (\textit{eXpress Data Path})}
\lstI
\end{frame}

% Segurança e monitoramento em IoT para mitigação e identificação de ataques
\begin{frame}{IoT (\textit{Internet of Things})}
    \begin{enumerate}
        \item Monitoramento
        \item Análise estatísica (Wi-Fi, LoraWan, NFC, etc.)
        \item Intervenção dinâmica (eBPF)
    \end{enumerate}
\end{frame}

% Básica, Exploratória, Bibliográfica e Experimental
\section{Pesquisa}

\begin{frame}{Objetivo}
    \begin{enumerate}
        \item Aprofundamento dos conhecimentos referentes ao arcabouço eBPF e pleno entendimento das oportunidades.
        
        \vspace{30pt}
        
        \item Experimentação em projetos de infraestruturas de IoT com incorporação do eBPF.
    \end{enumerate}
\end{frame}

\begin{frame}{Metodologia}
\begin{itemize}
		\item Básica
		\item Exploratória
		\item Bibliográfica
		\item Experimental
	\end{itemize} 
\end{frame}

\begin{frame}{Cronograma}
\begin{itemize}
		\item 6 meses de pesquisa.
		\item Laboratório de computação distibuída (LCD).
	\end{itemize}

	\begin{table}[]
        \centering
        \begin{tabular}{|l|c|c|c|c|c|c|}
        \hline
        \multicolumn{1}{|c|}{Atividade / Meses} & Jul & \multicolumn{1}{l|}{Ago} & \multicolumn{1}{l|}{Set} & \multicolumn{1}{l|}{Out} & \multicolumn{1}{l|}{Nov} & \multicolumn{1}{l|}{Dez} \\ \hline
        Pesquisa Bibliográfica                  & X     &                             &                               &                              &                               &                               \\ \hline
        \textit{Estruturação do projeto}        &       & X                           &                               &                              &                               &                               \\ \hline
        \textit{Planejamento da experimentação} &       & X                           &                               &                              &                               &                               \\ \hline
        \textit{Experimentação}                 &       &                             & X                             & X                            &                               &                               \\ \hline
        Tratamento de resultados                &       &                             &                               & X                            &                               &                               \\ \hline
        Relatório final                         &       &                             &                               & X                            & X                             &                               \\ \hline
        Revisão / correção do texto             &       &                             &                               &                              &                               & X                             \\ \hline
        \end{tabular}
        \label{table: Cronograma de pesquisa}
    \end{table}
\end{frame}

% Exploração e compreensão dos limites da tecnologia, bem como a projeção de uma infraestrutura que provê uma melhora na latência e/ou segurança
\begin{frame}{Resultados Esperados}
\begin{itemize}
		\item Autonomia para implementações envolvendo eBPF.
		
		\vspace{30pt}
		
		\item Protocolos que melhor usufruem do eBPF para melhoria em \textbf{latência} e \textbf{segurança}.
	\end{itemize} 
\end{frame}


% foto, email, linkedin e github
\begin{frame}{Contato}
{\centering Gabriel Marques de Melo \hspace{190pt}{\Large \color{black} \faLinux}}
    \vspace{15pt}
	\item {\huge \faGithub} \url{github.com/gabrielmmelo}
	\vspace{15pt}
	\item {\huge \faLinkedin} \url{linkedin.com/in/gabrielmmelo} 
	\vspace{15pt}
	\item {\huge \faEnvelopeO} \url{gabrielmarquesm@estudante.ufla.br} 
\vspace{15pt}
\begin{center}
    \small
    \textit{``Qualquer tecnologia suficientemente avançada \\
    é indistinta de magia``} \\
    \vspace{5pt}
    \footnotesize Arthur C. Clarke 
\end{center}	
	
\end{frame}
\end{document}
