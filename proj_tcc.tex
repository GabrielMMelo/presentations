\section{Contextualização}
\begin{frame}{}
	\begin{figure}
		\caption{Número de usuários ativos em 2019}
		\includegraphics[width = 0.65\linewidth]{img/lorem-picsum.jpg}\\
		\tiny Fonte: HootSuite, Abril de 2019
	\end{figure}
\end{frame}

% Modelos de negócios dependem cada vez mais da rede e demandam conexões de menor latência e maior segurança
\begin{frame}{}
	\begin{figure}
		\caption{}
		\includegraphics[width = 0.65\linewidth]{img/lorem-picsum.jpg}\\
		\tiny Fonte: HootSuite, Abril de 2019
	\end{figure}
\end{frame}

% No caso de redes IoT, onde muitos dispositivos trocam mensagens (que muitas vezes carregam informações sensíveis) entre si e com a internet, essa realidade é ainda mais complicada.
\begin{frame}{}
	\begin{figure}
		\caption{}
		\includegraphics[width = 0.65\linewidth]{img/lorem-picsum.jpg}\\
		\tiny Fonte: HootSuite, Abril de 2019
	\end{figure}
\end{frame}

% Surge a necessidade de dinamizar a rede inserindo programabilidade e a tornando flexível e mais segura.
\begin{frame}{Ideias gerais sobre transmissão de energia}
	\begin{itemize}
		\item<1-> A \textbf{transmissão} de energia é a parte do setor elétrico responsável por levar a energia elétrica das fontes de geração para as redes de distribuição. 
		\item De forma geral, durante a transmissão de energia, nenhuma carga é alimentada diretamente. Além disso, normalmente, a transmissão ocorre fora do perímetro urbano dos municípios.
		\end{itemize} 
\end{frame}


% eBPF + imagem do facebook e Cloudfare
\section{eBPF (extended Berkeley Packet Filter)}
\begin{frame}{eBPF - extended Berkeley Packet Filter}
\begin{itemize}
		\item Máquina virtual no kernel (\textit{segurança}).
		\item Interface dinâmica entre \textit{user space} e \textit{kernel space}.
		\item Opção flexível aos módulos atuais do kernel.
	\end{itemize} 
\begin{figure}
\centering
\begin{minipage}{.5\textwidth}
  \centering
  \includegraphics[width=.4\linewidth]{img/lorem-picsum.jpg}
  \captionof{figure}{A figure}
  \label{fig:test1}
\end{minipage}%
\begin{minipage}{.5\textwidth}
  \centering
  \includegraphics[width=.4\linewidth]{img/lorem-picsum.jpg}
  \captionof{figure}{Another figure}
  \label{fig:test2}
\end{minipage}
\end{figure}	
\end{frame}

% imagem com arquitetura da vm e workflow (passos e maps)
\begin{frame}{eBPF - extended Berkeley Packet Filter}
\begin{itemize}
		\item Máquina virtual no kernel (\textit{segurança}).
		\item Interface dinâmica entre \textit{user space} e \textit{kernel space}.
		\item Opção flexível aos módulos atuais do kernel.
	\end{itemize} 
\end{frame}

% eBPF e XDP. Figura da pilha de rede do kernel e explicação do funcionamento
\begin{frame}{eBPF \& XDP (\textit{eXpress Data Path})}
\begin{itemize}
		\item A \textbf{distribuição primária} leva a energia das subestações de distribuição até as subestações encontradas nas ruas e avenidas (transformadores de potência). Esse tipo de distribuição ocorre em média tensão, em sistemas conectados em delta.
		\item A \textbf{distribuição secundária} leva a energia dos transformadores até os consumidores finais. De forma geral, é feita em sistemas trifásicos em estrela aterrado, de baixa tensão.
	\end{itemize} 
\end{frame}

% código simples que realiza o descarte de todos os pacotes
\begin{frame}{eBPF \& XDP (\textit{eXpress Data Path})}
\begin{itemize}
		\item A \textbf{distribuição primária} leva a energia das subestações de distribuição até as subestações encontradas nas ruas e avenidas (transformadores de potência). Esse tipo de distribuição ocorre em média tensão, em sistemas conectados em delta.
		\item A \textbf{distribuição secundária} leva a energia dos transformadores até os consumidores finais. De forma geral, é feita em sistemas trifásicos em estrela aterrado, de baixa tensão.
	\end{itemize} 
\end{frame}

% Segurança e monitoramento em IoT para mitigação e identificação de ataques
\begin{frame}{eBPF \& XDP (\textit{eXpress Data Path})}
\begin{itemize}
		\item A \textbf{distribuição primária} leva a energia das subestações de distribuição até as subestações encontradas nas ruas e avenidas (transformadores de potência). Esse tipo de distribuição ocorre em média tensão, em sistemas conectados em delta.
		\item A \textbf{distribuição secundária} leva a energia dos transformadores até os consumidores finais. De forma geral, é feita em sistemas trifásicos em estrela aterrado, de baixa tensão.
	\end{itemize} 
\end{frame}

% Básica, Exploratória, Bibliográfica e Experimental
\section{Pesquisa}
\begin{frame}{Metodologia}
\begin{itemize}
		\item Para transmissão de energia, são adotados os seguintes níveis de tensão:
		\begin{itemize}
			\item Extra-baixa tensão: 48 V; 24 V e 12 V
			\item  Baixa tensão: 1.000 V; 760 V; 660 V; 440 V; 380 V; 220; 127 (FN) V; 115 (FN) V
			\item  Média tensão (ou alta tensão de distribuição):34,5 kV; 25,8 kV; 23 kV; 13,8 kV; 13,2 kV; 12,6 kV; 11,5 kV; 6,9 kV; 4,16 kV e 2,13 kV
			\item  Alta tensão (tensão de transmissão): 500 kV; 230 kV e 138 kV
			\item  Tensão de sub-transmissão: 69 kV
			\item  Extra-alta tensão: 600 kV (CC)
			\item  Extra-alta tensão: 750 kV
			\item  Ultra-alta tensão:800 kV
		\end{itemize}
	\end{itemize} 
\end{frame}

\begin{frame}{Cronograma}
\begin{itemize}
		\item 6 meses de pesquisa.
		\item Laboratório de computação distibuída.
	\end{itemize} 
	\begin{figure}
		\caption{}
		\includegraphics[width = 0.65\linewidth]{img/lorem-picsum.jpg}\\
		\tiny Fonte: HootSuite, Abril de 2019
	\end{figure}
\end{frame}

% Exploração e compreensão dos limites da tecnologia, bem como a projeção de uma infraestrutura que provê uma melhora na latência e/ou segurança
\begin{frame}{Resultados Esperados}
\begin{itemize}
		\item 6 meses de pesquisa.
		\item Laboratório de computação distibuída.
	\end{itemize} 
\end{frame}
